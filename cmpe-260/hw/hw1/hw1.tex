\documentclass{article}

\usepackage{graphicx}
\usepackage[a4paper, margin=1in]{geometry}
\usepackage{amsmath}
\usepackage{IEEEtrantools}
\usepackage{karnaugh-map}
\usepackage{circuitikz}
\usepackage{newtxtt}
\usepackage{newtxmath}

\graphicspath{{./assets/}}
\ctikzset{logic ports=ieee}

\title{DSD II Homework 1}
\author{Aden Perry}
\date{\today}

\begin{document}
\maketitle

\begin{enumerate}
	\item See attached VHDL.
	\item See attached VHDL.
	\item Treat x1, x2, and x3 as inputs. Load val1--8 with the correct outputs.

	      \begin{tabular}{|c|c|c|c|}
		      \hline
		      $x_1$ & $x_2$ & $x_3$ & $y$ \\
		      \hline
		      \hline
		      0     & 0     & 0     & 0   \\
		      \hline
		      0     & 0     & 1     & 0   \\
		      \hline
		      0     & 1     & 0     & 0   \\
		      \hline
		      0     & 1     & 1     & 1   \\
		      \hline
		      1     & 0     & 0     & 0   \\
		      \hline
		      1     & 0     & 1     & 1   \\
		      \hline
		      1     & 1     & 0     & 1   \\
		      \hline
		      1     & 1     & 1     & 1   \\
		      \hline
	      \end{tabular}

	      Ergo values should be mapped as (assuming upper input is selected on 0):

	      \begin{tabular}{|c|c|}
		      \hline
		      Port & Value \\
		      \hline
		      \hline
		      val1 & 0     \\
		      \hline
		      val2 & 0     \\
		      \hline
		      val3 & 0     \\
		      \hline
		      val4 & 1     \\
		      \hline
		      val5 & 0     \\
		      \hline
		      val6 & 1     \\
		      \hline
		      val7 & 1     \\
		      \hline
		      val8 & 1     \\
		      \hline
	      \end{tabular}

	\item Truth table:

	      \begin{tabular}{|c|c|c|c||c|c|}
		      \hline
		      $a_0$ & $a_1$ & $a_2$ & $a_3$ & $y_0$ & $y_1$ \\
		      \hline
		      \hline
		      1     & X     & X     & X     & 1     & 1     \\
		      \hline
		      0     & 1     & X     & X     & 1     & 0     \\
		      \hline
		      0     & 0     & 1     & X     & 0     & 1     \\
		      \hline
		      0     & 0     & 0     & 1     & 0     & 0     \\
		      \hline
		      0     & 0     & 0     & 0     & $a_0$ & $a_1$ \\
		      \hline
	      \end{tabular}

	      Which simplifies to:

	      \begin{tabular}{|c|c|c|c||c|c|}
		      \hline
		      $a_0$ & $a_1$ & $a_2$ & $a_3$ & $y_0$ & $y_1$ \\
		      \hline
		      \hline
		      1     & X     & X     & X     & 1     & 1     \\
		      \hline
		      0     & 1     & X     & X     & 1     & 0     \\
		      \hline
		      0     & 0     & 1     & X     & 0     & 1     \\
		      \hline
		      0     & 0     & 0     & X     & 0     & 0     \\
		      \hline
	      \end{tabular}

          In kmaps ($y_0$, $y_1$):

          \begin{karnaugh-map}(label=corner)[4][4][1][$a_1$][$a_0$][$a_3$][$a_2$]
              \maxterms{0,4,8,12}
              \implicant{1}{11}
              \implicant{3}{10}
              \autoterms[1]
          \end{karnaugh-map}
          \begin{karnaugh-map}(label=corner)[4][4][1][$a_1$][$a_0$][$a_3$][$a_2$]
              \maxterms{1,5,13,9,0,4}
              \implicant{3}{10}
              \implicant{12}{8}
              \autoterms[1]
          \end{karnaugh-map}

          $y_0 = a_0 + a_1$ (1 gate); $y_1 = a_0 + \bar {a_0} \bar {a_1} a_2$ (5 gates).

          \begin{circuitikz}
              \node (a0) at (0,0) {$a_0$};
              \node (a1) at (1,0) {$a_1$};
              \node (a2) at (2,0) {$a_2$};
              \node (a3) at (3,0) {$a_3$};

              \draw (8,1) node[or port] (or0) {};

              \draw (0,5) node[not port, rotate = 90] (not0) {};
              \draw (1,5) node[not port, rotate = 90] (not1) {};
              \draw (5,7) node[and port, number inputs = 3] (and0) {};
              \draw (8,3) node[or port] (or1) {};

              \draw (a0) |- node[circ, midway] {} (or0.in 1);
              \draw (a1) |- node[circ, midway] {} (or0.in 2);

              \draw (a0) |- (not0.in);
              \draw (a1) |- (not1.in);
              \draw (not0.out) |- (and0.in 1);
              \draw (not1.out) |- (and0.in 2);
              \draw (a2) |- (and0.in 3);

              \draw (a0) |- node[circ, midway] {} (or1.in 2);
              \draw (and0.out) |- (or1.in 1);

              \node (y0) at (10,1) {$y_0$};
              \node (y1) at (10,3) {$y_1$};
          \end{circuitikz}
\end{enumerate}

\end{document}
