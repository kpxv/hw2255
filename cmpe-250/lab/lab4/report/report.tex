\documentclass[CMPE]{KGCOEReport}

\usepackage{graphicx}
\usepackage{newtxtext, newtxmath}
\usepackage{parskip}
\usepackage{karnaugh-map}
\usepackage{float}
\usepackage{pdfpages}

\graphicspath{{./assets/}}

\newcommand{\classCode}{CMPE 250}
\newcommand{\name}{Aden Perry}
\newcommand{\LabSectionNum}{L1}
\newcommand{\LabInstructor}{Chongzhou Fang}

\newcommand{\TAs}{Eddie Satowski \\ Khoi Pham}
\newcommand{\LectureSectionNum}{01}
\newcommand{\LectureInstructor}{Chongzhou Fang}
\newcommand{\exerciseNumber}{4}
\newcommand{\exerciseDescription}{Iteration and Subroutines}
\newcommand{\dateDone}{2026-02-05}
\newcommand{\dateSubmitted}{2025-02-13}

\begin{document}

\maketitle

\section*{Abstract}

The purpose of this exercise was to design, implement, and test an unsigned
division algorithm written in armasm for the ARMv6-M architecture. The
algorithm used a subtraction loop to calculate the result of a division
operation. A library was imported to test the algorithm. Two word variables
and a word array were created to store results and interface between the the
test functions and the algorithm. All tests passed; the exercise was
successful.

\section*{Procedure}

An algorithm was designed and tested using the GNU Assembler. Register R2 was
pushed to the stack and initialized to 0; it was used to track the quotient.
\texttt{CMP} and conditional branching were used to subtract the divisor from
the dividend until the dividend was less than the divisor. Each iteration,
the quotient was incremented by 1. Upon loop exit, the quotient was moved into
register R0, the \texttt{ADDS} instruction was used to clear the CARRY flag,
and register R2 was popped from the stack. Simple error checking occured at the
entrance to the DIVU algorithm to ensure no division by 0 could occur. Instead,
input values were preserved and the CARRY flag was set.

After the algorithm was written, it was ported to Keil ARM Assembly for the
purpose of importing and running the test library. A Keil MDK-ARM project was
created. A template Assembly program was loaded. The DIVU algorithm was added
to the end of the \texttt{MyCode} area.

Two variables, \texttt{P} and \texttt{Q}, were created in the \texttt{MyData}
area for the purpose of communicating with the test library. An array was
created to store results. A test loop was created at the start of the
\texttt{MyCode} area. The loop loaded the test case from the library into
variables \texttt{P} and \texttt{Q}. It then called the \texttt{DIVU}
algorithm. It stored the results into the variables, and used the library to
test them.

The program was translated, built, and debugged. Using the debugger, the tests
were confirmed to pass by viewing the register contents after all test cases
were run.

\section*{Results}

The test library counted the number of tests that were not passed in register
R6. It also tracked which was the last run test in R7; the last of the test
had the value 0xA0. Thus, the expected register values were an R6 of 0 and an
R7 of 0xA0. The actual results are presented in Figure \ref{fig:register}.

\begin{figure}[H]
	\center
	\includegraphics{registers}
	\caption{The final register values}
	\label{fig:register}
\end{figure}

As seen in Figure \ref{fig:register}, both registers matched the expected
values.

The image symbol table of the program is given in Table \ref{tab:sym},
including all local symbols and relevant global symbols.

\begin{table}[H]
	\center
	\caption{The image symbol table}
	\begin{tabular}{|c|c|c|c|c|}
		\hline
		Symbol Name             & Value      & Type       & Size & Object (Section)                   \\
		\hline
		\hline
		RESET                   & 0x00000000 & Section    & 192  & lab4.o(RESET)                      \\
		\hline
		Exercise04\_Lib.s       & 0x00000000 & Number     & 0    & exercise04\_lib.o ABSOLUTE         \\
		\hline
		dc.s                    & 0x00000000 & Number     & 0    & dc.o ABSOLUTE                      \\
		\hline
		lab4.s                  & 0x00000000 & Number     & 0    & lab4.o ABSOLUTE                    \\
		\hline
		Exercise04\_Lib         & 0x000000c0 & Section    & 420  & exercise04\_lib.o(Exercise04\_Lib) \\
		\hline
		InitData                & 0x000000c1 & Thumb Code & 6    & exercise04\_lib.o(Exercise04\_Lib) \\
		\hline
		LoadData                & 0x000000c7 & Thumb Code & 38   & exercise04\_lib.o(Exercise04\_Lib) \\
		\hline
		TestData                & 0x000000ed & Thumb Code & 44   & exercise04\_lib.o(Exercise04\_Lib) \\
		\hline
		MyCode                  & 0x00000264 & Section    & 160  & lab4.o(MyCode)                     \\
		\hline
		Reset\_Handler          & 0x00000265 & Thumb Code & 50   & lab4.o(MyCode)                     \\
		\hline
		RegInit                 & 0x00000297 & Thumb Code & 52   & lab4.o(MyCode)                     \\
		\hline
		DIVU                    & 0x000002cb & Thumb Code & 28   & lab4.o(MyCode)                     \\
		\hline
		.ARM.\_\_at\_0x1FFFFC00 & 0x1ffffc00 & Section    & 256  & lab4.o(.ARM.\_\_at\_0x1FFFFC00)    \\
		\hline
		MyData                  & 0x1ffffd00 & Section    & 208  & lab4.o(MyData)                     \\
		\hline
		P                       & 0x1ffffd00 & Data       & 4    & lab4.o(MyData)                     \\
		\hline
		Q                       & 0x1ffffd04 & Data       & 4    & lab4.o(MyData)                     \\
		\hline
		Results                 & 0x1ffffd08 & Data       & 200  & lab4.o(MyData)                     \\
		\hline
	\end{tabular}
	\label{tab:sym}
\end{table}

According to Table \ref{tab:sym}, the \texttt{Reset\_Handler} symbol (i.e.\ the
main program) occupies 50 bytes. This is confirmed by subtracting the start
address of \texttt{Reset\_Handler}, 0x265, from the start address of the
following symbol \texttt{RegInit}, 0x297. This does not include the content of
\texttt{DIVU} or \texttt{RegInit} — combined, they make up the contents of the
\texttt{MyCode} area, which is clearly seen to have a size of 160 bytes.

The ExerciseLib library code is clearly shown to have a byte size of 420. There
are 3 variables: \texttt{P}, \texttt{Q}, and \texttt{Results}. By summing their
sizes, the total byte size of the variables can be found. The sum reveals a
byte size of 208. This is confirmed by checking the size of the \texttt{MyData}
area, which is the same.

The stack is contained in the area \texttt{.ARM.\_\_at\_0x1FFFFC00}, which has
a size of 256. This is confirmed by viewing \texttt{SSTACK\_SIZE} EQUate inside
the assembly file, which is used to assign the stack memory. It is equal to
0x100; thus, the stack has a byte size of 256.

\section*{Conclusion}

The object of this exercise was to design, implement, and test a DIVU function
using an imported library. This was accomplished through the usage of a
template to initialize the program and by adding subroutines to perform the
logic. Division was done through pure subtraction; this was simple to
implement, but slow for large numbers. In the future, bit shifting might be
considered as an alternative. The use of branching and of variables can and
should be used for any future assembly programs, especially if complex.
Variables are useful for storing values outside the registers; branching is
required for complex logic.

\includepdf[pages=-]{assets/prelab}

\end{document}
