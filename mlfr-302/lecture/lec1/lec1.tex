\documentclass{article}

\usepackage[a4paper, margin=1in]{geometry}
\usepackage[french]{isodate}
\usepackage{setspace}

\doublespacing

\title{Lecture 1}
\author{Aden Perry}
\date{\today}

\begin{document}
\maketitle

\section{Les mots}

\begin{figure}[ht]
	\center
	\begin{tabular}{|r|l|}
		\hline
		Français        & English        \\
		\hline
		\hline
		l'âge de pierre & Stone Age      \\
		\hline
		cuir            & leather        \\
		\hline
		afin            & so, so that    \\
		\hline
		mur             & wall           \\
		\hline
		gauloises       & Gaullish       \\
		\hline
		bouillie        & mush           \\
		\hline
		orge            & barley         \\
		\hline
		blé             & wheat          \\
		\hline
		agrémenter      & to liven up    \\
		\hline
		miel            & honey          \\
		\hline
		dater de        & date back to   \\
		\hline
		emporte-pièces  & cookie cutters \\
		\hline
		saisonnier      & seasonal       \\
		\hline
		betterave       & beetroot       \\
		\hline
	\end{tabular}
\end{figure}

\section{Les questions}

\begin{enumerate}
	\item Qui sont les premiers à avoir découvert la farine et fabriqué les
	      premières galettes de pain selon le texte?

	      Les hommes de l'âge de pierre les ont découvert.

	\item Quelles civilisations ont réalisé des galettes à base de bouillie de
	      céréales, agrémentées de graines de pavot, d'anis, de fenouil, et
	      parfumées au miel et à la coriandre?

	      Les civilisations grecques, romaines, et gauloises en ont réalisé.

	\item Quand les premières pâtisseries réalisées avec du miel ont-elles été
	      développées, et par quelle civilisation?

	      Les civilisations sumériennes les ont développées en 4000 avant
	      Jésus-Christ.

	\item Quelle corporation a été fondée à Rome au 4ième siècle avant
	      Jésus-Christ, et quel était son objectif?

	      La corporation est s'appellé Pastillariorum. Elle a eu l'objectif de
	      constiuer des lois communes.

	\item À quelle époque et sous quelle forme les emporte-pièces ont-ils été
	      utilisés pour la première fois, et quels étaient les motifs
	      populaires?

	      Ils est utilisés sous la forme de découpoirs en bois à l'epoque
	      d'Égypte acnienne.

	\item Quand le sucre de canne est-il apparu en France, et comment a-t-il
	      été introduit?

	      Il a été y introduit au 10ième siécle avant Jésus-Christ par les
	      Croisés.

	\item Quelle période historique est associée à la prédominance des
	      préparations salées en pâtisserie en Europe?

	      Le période entre 500 et 1500 ans aprés Jésus-Christ y est associée.

	\item Qui est considéré comme le premier vrai théoricien de la pâtisserie,
	      et quelles techniques a-t-il découvertes avant de créer les premières
	      pièces montées?

	      Antonin Carême est the premier vria théoricien qui a découvert les
	      techniques du soufflet.

\end{enumerate}

\section{La recherche}

\begin{enumerate}
	\item \textbf{Soufflet:}

	      Il est crées avec un combinaison des blanc d'œufs et une crème
	      pâtisserie ou une purée. Il peut-etre salé ou sucré ; les soufflets
	      salé est crée avec du légumes ou du fromages, et les soufflets sucrés
	      est crée avec des fruits, du chocolat, ou de la confiture.

	\item \textbf{Savarin:}

	      Un savarin est une variante de baba au rhum. C'est un petit gateau
          à la levure qui est trempé avec alcool. Il a la forme d'un tore.

	\item \textbf{Croque-en-bouche:}

	      Une croque-en-bouche est un collection des pâte à choux, lié ensemble
          avec caramel. Les pâte à choux est crée avec des œufs, du beurre, de
          l'eau, et de la farine. Quand ils est mis dans une croque-en-bouche,
          ils est se rempli souvent avec crème.

\end{enumerate}

\end{document}
